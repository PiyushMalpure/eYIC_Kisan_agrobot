%%%%%%%%%%%%  Generated using docx2latex.com  %%%%%%%%%%%%%%

%%%%%%%%%%%%  v2.0.0-beta  %%%%%%%%%%%%%%

\documentclass[12pt]{article}
\usepackage{amsmath}
\usepackage{latexsym}
\usepackage{amsfonts}
\usepackage[normalem]{ulem}
\usepackage{soul}
\usepackage{array}
\usepackage{amssymb}
\usepackage{extarrows}
\usepackage{graphicx}
\usepackage[backend=biber,
style=numeric,
sorting=none,
isbn=false,
doi=false,
url=false,
]{biblatex}\addbibresource{bibliography.bib}

\usepackage{subfig}
\usepackage{wrapfig}
\usepackage{wasysym}
\usepackage{enumitem}
\usepackage{adjustbox}
\usepackage{ragged2e}
\usepackage[svgnames,table]{xcolor}
\usepackage{tikz}
\usepackage{longtable}
\usepackage{changepage}
\usepackage{setspace}
\usepackage{hhline}
\usepackage{multicol}
\usepackage{tabto}
\usepackage{float}
\usepackage{multirow}
\usepackage{makecell}
\usepackage{fancyhdr}
\usepackage[toc,page]{appendix}
\usepackage[hidelinks]{hyperref}
\usetikzlibrary{shapes.symbols,shapes.geometric,shadows,arrows.meta}
\tikzset{>={Latex[width=1.5mm,length=2mm]}}
\usepackage{flowchart}\usepackage[paperheight=11.69in,paperwidth=8.27in,left=1.0in,right=1.0in,top=1.0in,bottom=1.0in,headheight=1in]{geometry}
\usepackage[utf8]{inputenc}
\usepackage[T1]{fontenc}
\TabPositions{0.5in,1.0in,1.5in,2.0in,2.5in,3.0in,3.5in,4.0in,4.5in,5.0in,5.5in,6.0in,}

\urlstyle{same}


 %%%%%%%%%%%%  Set Depths for Sections  %%%%%%%%%%%%%%

% 1) Section
% 1.1) SubSection
% 1.1.1) SubSubSection
% 1.1.1.1) Paragraph
% 1.1.1.1.1) Subparagraph


\setcounter{tocdepth}{5}
\setcounter{secnumdepth}{5}


 %%%%%%%%%%%%  Set Depths for Nested Lists created by \begin{enumerate}  %%%%%%%%%%%%%%


\setlistdepth{9}
\renewlist{enumerate}{enumerate}{9}
		\setlist[enumerate,1]{label=\arabic*)}
		\setlist[enumerate,2]{label=\alph*)}
		\setlist[enumerate,3]{label=(\roman*)}
		\setlist[enumerate,4]{label=(\arabic*)}
		\setlist[enumerate,5]{label=(\Alph*)}
		\setlist[enumerate,6]{label=(\Roman*)}
		\setlist[enumerate,7]{label=\arabic*}
		\setlist[enumerate,8]{label=\alph*}
		\setlist[enumerate,9]{label=\roman*}

\renewlist{itemize}{itemize}{9}
		\setlist[itemize]{label=$\cdot$}
		\setlist[itemize,1]{label=\textbullet}
		\setlist[itemize,2]{label=$\circ$}
		\setlist[itemize,3]{label=$\ast$}
		\setlist[itemize,4]{label=$\dagger$}
		\setlist[itemize,5]{label=$\triangleright$}
		\setlist[itemize,6]{label=$\bigstar$}
		\setlist[itemize,7]{label=$\blacklozenge$}
		\setlist[itemize,8]{label=$\prime$}



 %%%%%%%%%%%%  Header here  %%%%%%%%%%%%%%


\pagestyle{fancy}
\fancyhf{}
\chead{ 
\vspace{\baselineskip}
}
\renewcommand{\headrulewidth}{0pt}
\setlength{\topsep}{0pt}\setlength{\parindent}{0pt}

 %%%%%%%%%%%%  This sets linespacing (verticle gap between Lines) Default=1 %%%%%%%%%%%%%%


\renewcommand{\arraystretch}{1.3}


%%%%%%%%%%%%%%%%%%%% Document code starts here %%%%%%%%%%%%%%%%%%%%



\begin{document}
\begin{adjustwidth}{0.5in}{0.52in}
\begin{Center}
{\fontsize{18pt}{21.6pt}\selectfont \textbf{IDEA PROPOSAL FORMAT}\par}
\end{Center}\par

\end{adjustwidth}

\setlength{\parskip}{8.04pt}
\begin{adjustwidth}{0.5in}{0.52in}
\begin{Center}
{\fontsize{18pt}{21.6pt}\selectfont \textbf{e-Yantra Ideas Competition 2019-20}\par}
\end{Center}\par

\end{adjustwidth}

\begin{adjustwidth}{0.5in}{0.52in}
{\fontsize{14pt}{16.8pt}\selectfont \textbf{\uline{Project Name:}}\par}\tab \par

\end{adjustwidth}

\begin{adjustwidth}{0.5in}{0.52in}
Kisan - Agriculture robot for precise seeding and controlling greenhouse parameter\par

\end{adjustwidth}

\begin{adjustwidth}{0.5in}{0.52in}
{\fontsize{14pt}{16.8pt}\selectfont \textbf{\uline{Introduction/Motivation:}}\par}\par

\end{adjustwidth}

\begin{itemize}
	\item Growing demand of food is putting tremendous pressure on current farms. Farmers are trying new techniques to improve farm’s yield. \par

	\item Studies are done on how to utilize minimum resources to get the maximum input but currently no affordable technologies are available.\par

	\item In the current market there are equipment for precise farming but these equipments are heavy and are not affordable for small farmers. \par

	\item With technology like IOT and Embedded Robotics we can provide an affordable solution to the small farmers. \par

	\item Problems faced in traditional farming are 
\end{itemize}\par

\begin{itemize}
	\item Overuse of fertilizers, pesticides or insecticides.\par

	\item Lack of manpower which is must needed in task like seeding.\par

	\item Lack of control over parameters like humidity, soil moisture and temperature.\par


\end{itemize}
	\item When we consider a state like Maharashtra, there is a scarcity of resources but with the help of robotics we can help in reducing in consumption of resources like seeds, water, fertilizers etc. \par

\begin{adjustwidth}{0.5in}{0.52in}
\begin{justify}
This inspired us to work on a project which would benefit our farmers.
\end{justify}\par

\end{adjustwidth}

\begin{adjustwidth}{0.5in}{0.52in}
{\fontsize{14pt}{16.8pt}\selectfont \textbf{\uline{Market Research / Literature Survey:}}\par}\par

\end{adjustwidth}

\begin{adjustwidth}{0.5in}{0.52in}
\begin{justify}
Existing systems for greenhouse monitor are used to monitor various greenhouse parameters like Temperature, Humidity and Soil Moisture. These systems will display the data locally as well send a message to the owner in case any parameter in not within limits [3].
\end{justify}\par

\end{adjustwidth}

\begin{adjustwidth}{0.5in}{0.52in}
\begin{justify}
Major problem faced while moving a rover on uneven surface is that they need to overcome the obstacles on the surface. So NASA has used rocker bogie mechanism for traversing on uneven surface and overcome any obstacles present [1][2].
\end{justify}\par

\end{adjustwidth}

\begin{adjustwidth}{0.5in}{0.52in}
\begin{justify}
From our visits to farms and agricultural college we understood that problems faced by farmers are availability of labour and expensive technologies present in the market. The cost of seeds is high, so they need to avoid wastage of seeds to increase their profits.
\end{justify}\par

\end{adjustwidth}

\begin{adjustwidth}{0.5in}{0.52in}
{\fontsize{14pt}{16.8pt}\selectfont \textbf{\uline{Hardware requirements:}}\par}\par

\end{adjustwidth}

\begin{itemize}
	\item Servo motors for seeding mechanism\par

	\item DC motors with encoders to calculate distance traversed\par

	\item DC motors without encoders for traversing \par

	\item Motor Driver for DC motor\par

	\item PWM driver for servo motor\par

	\item Arduino mega for controlling motors and accept input from sensors\par

	\item Raspberry Pi for High Computation\par

	\item Wi-Fi Shield for IOT\par

	\item GSM Module for sending message to greenhouse owner\par

	\item Suction Pump for seeding\par

	\item Humidity and temperature sensor to monitor temperature and humidity in greenhouse\par

	\item Soil Moisture Sensor to monitor soil moisture in greenhouse
\end{itemize}\par

\begin{adjustwidth}{0.5in}{0.52in}
{\fontsize{14pt}{16.8pt}\selectfont \textbf{\uline{Software requirements:}}\par}\par

\end{adjustwidth}

\begin{enumerate}[label*={\fontsize{12pt}{12pt}\selectfont \arabic*.}]
	\item Arduino IDE: For programming Atmega2560\par

	\item Eagle: For PCB design\par

	\item Python for Image Processing on Raspberry PI
\end{enumerate}\par


\vspace{\baselineskip}

\vspace{\baselineskip}
\begin{adjustwidth}{0.5in}{0.52in}
{\fontsize{14pt}{16.8pt}\selectfont \textbf{\uline{Implementation:}}\par}\par

\end{adjustwidth}

\begin{adjustwidth}{0.5in}{0.52in}
\begin{justify}
Main challenge faced for robots is traversing, which we are solving by using a mechanism called Rocker Bogie Mechanism shown in figure 1. The rocker-bogie suspension is a mechanism that, along with a differential, enables a six-wheeled vehicle to passively keep all six wheels in contact with a surface even when driving on severely uneven terrain. There are two key advantages to this feature. The first advantage is that the wheels pressure on the ground will be equilibrated. This is extremely important in soft terrain where excessive ground pressure can result in the vehicle sinking into the driving surface. The second advantage is that while climbing over hard, uneven terrain, all six wheels will nominally remain in contact with the surface and under load, helping to propel the vehicle over the terrain [2]. Precision in traversing is obtained by using encoders used to calculate distance. 
\end{justify}\par

\end{adjustwidth}



%%%%%%%%%%%%%%%%%%%% Figure/Image No: 1 starts here %%%%%%%%%%%%%%%%%%%%

\begin{figure}[H]
	\begin{Center}
		\includegraphics[width=4.92in,height=2.21in]{./media/image1.png}
	\end{Center}
\end{figure}


%%%%%%%%%%%%%%%%%%%% Figure/Image No: 1 Ends here %%%%%%%%%%%%%%%%%%%%

\par

\setlength{\parskip}{9.96pt}
\begin{adjustwidth}{0.5in}{0.52in}
\begin{Center}
{\fontsize{9pt}{10.8pt}\selectfont \textit{\textcolor[HTML]{1F497D}{Figure 1:Rocker Bogie [1]}}\par}
\end{Center}\par

\end{adjustwidth}

\begin{adjustwidth}{0.5in}{0.52in}
\begin{justify}
Traversing for a robot in an agricultural field becomes difficult because it looses track, due to numerous obstacles and doesn’t have any guiding line to get it back on track. The field being uneven makes it very hard for moving in a precise motion. 
\end{justify}\par

\end{adjustwidth}



%%%%%%%%%%%%%%%%%%%% Figure/Image No: 2 starts here %%%%%%%%%%%%%%%%%%%%

\begin{figure}[H]
	\begin{Center}
		\includegraphics[width=6.25in,height=2.17in]{./media/image2.png}
	\end{Center}
\end{figure}


%%%%%%%%%%%%%%%%%%%% Figure/Image No: 2 Ends here %%%%%%%%%%%%%%%%%%%%

\par

\begin{adjustwidth}{0.5in}{0.52in}
\begin{Center}
{\fontsize{9pt}{10.8pt}\selectfont \textit{\textcolor[HTML]{1F497D}{Figure 2 Block Diagram}}\par}
\end{Center}\par

\end{adjustwidth}


\vspace{\baselineskip}
\begin{adjustwidth}{0.5in}{0.52in}
\begin{justify}
Figure 2 shows the block diagram of our system. Different sensors mounted on robot like Soil Moisture Sensor, Temperature and Humidity sensor will collect data about different greenhouse parameters and store it in the database. If any parameter drops below threshold the farmer will be informed so that they can take corrective action. The robot will traverse in the greenhouse and perform seeding with the help of arm mechanism.
\end{justify}\par

\end{adjustwidth}

\begin{adjustwidth}{0.5in}{0.52in}
\begin{justify}
Robot is going to perform seeding in the following manner:
\end{justify}\par

\end{adjustwidth}

\begin{adjustwidth}{0.5in}{0.52in}
\begin{justify}
A mechanism having suction pump with pointed nozzle having length appropriate enough to plant the seeds at required depth. The above mechanism will be mounted on servo which will pick the required mechanism as given by the controller.\textcolor[HTML]{0070C0}{ Servo motors help control the movement of arm precisely. Nozzle attached to the suction pump will pick a seed from seed tray placed on rover and release at certain depth in soil.}
\end{justify}\par

\end{adjustwidth}

\begin{adjustwidth}{0.5in}{0.52in}
\begin{justify}
Various parameters like Soil Moisture, Humidity and Temperature of greenhouse is monitored with the capacitive soil moisture sensor and amt1001 sensor, data gathered by the sensors will be delivered to the user via internet, these data will help in continuous monitoring, control signals will be taken from the users via internet [3]. Most of the times it is difficult to a good internet connection in agriculture fields so system will also have a GSM module which will send a direct message to user if one of three parameters fall below the predefined condition.
\end{justify}\par

\end{adjustwidth}


\vspace{\baselineskip}
\setlength{\parskip}{8.04pt}
\begin{adjustwidth}{0.5in}{0.52in}
{\fontsize{14pt}{16.8pt}\selectfont \textbf{\uline{Feasibility:}}\par}\par

\end{adjustwidth}

\begin{adjustwidth}{0.5in}{0.52in}
According to our discussion with farmers, the major problems they face are labour costs and cost seeding. So we focussed on solving these problems. The material used for structure is PVC which is easily available in all hardware stores. So in case of any damage to the structure, it can be repaired easily. Use of PVC has also reduced the cost of structure significantly. The rover should be cheaper but at the same must be able to perform its tasks efficiently. Precise seeding is helpful as it helps to increase yield of crop. It will help in reducing cost of seeding and labour.\par

\end{adjustwidth}

\begin{adjustwidth}{0.5in}{0.52in}
Sensors like temperature and humidity sensor and soil moisture sensor will help the farmer to monitor the greenhouse without actually visiting it. They just have to visit the greenhouse in case any greenhouse parameters(Temperature, Humidity and Soil Moisture) is not within the required limits.\par

\end{adjustwidth}

\begin{adjustwidth}{0.5in}{0.52in}
{\fontsize{14pt}{16.8pt}\selectfont \textbf{\uline{References:}}\par}\par

\end{adjustwidth}

\begin{adjustwidth}{0.5in}{0.52in}
\begin{justify}
 [1] Hervé Hacot1, Steven Dubowsky, Philippe Bidaud, \textcolor[HTML]{00796B}{$"$ }ANALYSIS AND SIMULATION OF A ROCKER-BOGIE EXPLORATION ROVER\textcolor[HTML]{00796B}{$"$ }, Department of Mechanical Engineering, Massachusetts Institute of Technology, Cambridge.
\end{justify}\par

\end{adjustwidth}

\begin{adjustwidth}{0.5in}{0.52in}
\begin{justify}
[2] Brian D. Harrington and Chris Voorhees,\textcolor[HTML]{00796B}{$"$ }The Challenges of Designing the Rocker-Bogie Suspension for the Mars Exploration Rover\textcolor[HTML]{00796B}{$"$ } , NASA.
\end{justify}\par

\end{adjustwidth}

\setlength{\parskip}{18.96pt}
\begin{adjustwidth}{0.5in}{0.52in}
[3] Ravi Kishore Kodali, Vishal Jain and Sumit Karagwal: IoT based Smart Greenhouse\par

\end{adjustwidth}


\printbibliography
\end{document}